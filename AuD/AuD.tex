\documentclass{book}
\usepackage[ngerman]{babel}
\begin{document}
\begin{titlepage}
    \centering
	{\LARGE \textsc{Hochschule Darmstadt}\par}
	\vspace{1cm}
	{\Large \textsc{Zusammenfassung: 1. Semester}\par}
	\vspace{1.5cm}
	{\huge\bfseries Algorithmen und Datenstrukturen\par}
	\vspace{2cm}
	{\Large\itshape Leonhard Breuer\par}

% Bottom of the page
	{\large \today\par}
\end{titlepage}
\tableofcontents

\chapter{Der Begriff Algorithmus}
\paragraph{Den Begriff Algorithmus} gibt es bereits seit dem Mittelalter.
\paragraph{Das Konzept Algorithmus} war bereits den alten Griechen bekannt.

\section{Sieb des Erasthostenes}
Das S.d.E. ermöglicht das Finden aller Primzahlen bis zu einer gewählten maximalen Zahl $n$.
\paragraph*{Ablauf des Algorithmus} ist wie folgt
\begin{enumerate}
	\item Schreibe alle Zahlen von 1 bis $n$ auf.
	\item Für alle Zahlen $i$ von 2 bis $\sqrt{n}$: Alle Vielfachen von $i$ streichen.
	\item Schreibe alle nicht gestrichenen Zahlen herraus.
\end{enumerate}
\section{Algorithmus des Euklied (ggT)}
Der Algorthimus des Euklied dient zum Herausfinden des \textit{größten gemeinsamen Teilers} zweier Zahlen.
\paragraph*{Ablauf des Algorithmus} ist wie folgt. \\
Ausgehend von den Zahlen $m$ und $m$ < 0 sowie $r$ welche den Rest der (ganzzahligen) Division repräsentiert.
\begin{enumerate}
	\item Teile $m$ durch $n$ mit Rest $r$ (es ergibt sich $r \geq n$).
	\item Wenn $r = 0$, fertig mit Ergebnis $n$.
	\item Ersetze $m$ mit $n$ und $n$ mit $r$ und mache mit \textit{1.} weiter.
\end{enumerate}
\section{Algorism}
Das Wort \textit{Algorism} beschreibt das Addieren im Stellenwertsystem.
\paragraph*{Ablauf des Algorithmus} ist wie folgt
\begin{enumerate}
	\item Schreibe die zu addierenden Zahlen rechtsbündig untereinenander.
	\item Beginne ganz rechts mit der ersten Zahl.
	\item Addiere die über dem Strich stehenden Ziffern der aktuellen Stelle zusammen.
	\item Schreibe das Ergebnis unter den Strich
	\item Setze diese Operationen für alle restlichen Stellen (beider Zahlen) fort.
\end{enumerate}
\chapter{Graphen und Bäume}
Graphen und Bäume werden in der Informatik oftmals zur Beschreibung und Vereinfachung von  Problemen benötigt.
\paragraph*{Ein Graph} besteht aus \texttt{Knoten (nodes)} und \texttt{Kanten (edges)}.
\subparagraph*{Knoten} bilden die Menge \texttt{V} von Objekten \textit{v}.
\subparagraph*{Kanten} bilden die Menge \texttt{E} von Tupeln \textit{e = (v, w)} mit $v,w \in V$. 
\section{Graphen}
\subsection{Pfade} Man unterscheidet zwischen \texttt{offenen} und \texttt{geschlossenen} Pfaden.
\begin{description}
	\item[offener Pfad] Start- und Endpunkt sind unterschiedlich
	\item[geschlossener Pfad]  Führt am Ende an der Startpunkte zurück (Schleife)
\end{description}
\subsection{Annotationen}
Annotationen können anliegen, an
\begin{itemize}
	\item \texttt{Knoten}: Als Knotengewicht
	\item \texttt{Kanten}: Als Kantengewicht
\end{itemize}
Annotationen, seltener Gewichtungen, ermöglichen es Algorithmen, bzw. den kürzesten Weg zu ermitteln.
\subsection{Grad eines Knotens}
\begin{description}
	\item[Bei gerichteten Kanten] unterscheidet man Aus- und Eingangsgrad.
	\item[Bei ungerichteten Kanten] spricht man nur vom "Grad". 
\end{description}
Vereinfacht stellt der "Grad" eines Knoten die Anzahl der ein-/ausgehenden Verbindungen des Knoten(im Falle eines gerichteten Graphen) 
und der Gesamtanzahl der Verbindungen des Knoten (im Falle eines ungerichteten Graphens) dar.
\subsection{Gerichtete Graphen}
\begin{itemize}
	\item Alle Pfade sind \textit{offen}.
	\item auch \textit{Digraph} oder \textit{directed graph}
\end{itemize}
\subsection{Gewichtete Graphen}
Ein Graph ist gewichtet, sobald er über Annotationen verfügt.
\subsection{Dependenzgraph (DAG)}
Bei einem Dependenzgraphen beschreiben die Kanten, welche Tätigkeiten/Aufgaben vor 
einer anderen erfüllt werden müssen.
Enthält der gerichtete Graph hierbei keine Zyklen, so spricht man vom \textit{directed acyclic graph} oder kurz \textit{DAG}.
\subsection{Ungerichteter Graph / Cliquen}
Eine \textit{ungerichtete} Kante verhält sich wie zwei gerichtete Kanten. (eine hin und eine andere zurück)

\paragraph{Zusammenhang} Wenn von einem Element aus, alle anderen Elemente (Knoten) in einem Graphen erreichbar sind,
dann spricht man von einem zusammenhängenden Graphen. Gruppen, welche zwischen sich verbunden aber nicht mit anderen Gruppen verbunden sind, nennt man \textit{Zusammenhangskomponenten}
\subsection{Multigraph} 
Gibt es mehrere Kanten zwischen zwei Knoten (sowohl gerichtet als auch ungerichtet), so spricht man von einem
Multigraph.
\section{Bäume}
Ein Baum ist \textit{zusammenhängender} \textit{ungerichteter} Graph ohne \textit{geschlossenen} Pfad.
\subsection{Knoten und Blätter}
\begin{description}
	\item[Knoten mit Grad 1]  werden als Blatt (oder engl. Leaf) bezeichnet.
	\item[Andere Knoten] werden als innere Knoten bzw. \textit{inner leaf} bezeichnet. 
	\item[Elter(n)] ist der übergeordnete Knoten
	\item[Kind] ist der untergeordnete Knoten 
\end{description}
\subsection{Bäume - Höhe, Grad}
\begin{description}
	\item[Grad] Der Grad eines Baumes ist immer der höchste Grad eines Knotens.
	\item[Höhe] maximale Pfadlänge
\end{description}
\end{document}